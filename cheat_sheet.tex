\documentclass[10pt,landscape,a4paper]{article}

%*****************************************************************************
% Copyright (C) 2014 Volker Braun <vbraun.name@gmail.com> (git-cheat-sheet.tex)
% Copyright (C) 2015 Sébastien Labbé (MCFAlgorithms cheat sheets)
% Creative Commons Attribution-Share Alike 3.0 License.
%*****************************************************************************

\usepackage{multicol}
\usepackage{calc}
\usepackage{ifthen}
\usepackage[landscape]{geometry}
\usepackage{amsmath,amsthm,amsfonts,amssymb}
\usepackage{color,graphicx,overpic}
\usepackage{hyperref}
\usepackage[T1]{fontenc}
\usepackage{lmodern}         % Polices vectorielles
\usepackage[utf8]{inputenc}  % Codage UNICODE (UTF-8)
\usepackage{cite}

\pdfinfo{
  /Title (MCF Algorithms cheat sheet.pdf)
  /Creator (TeX)
  /Producer (pdfTeX 1.40.0)
  /Author (Sébastien Labbé)
  /Subject (Example)
  /Keywords (pdflatex, latex,pdftex,tex)}

% This sets page margins to .5 inch if using letter paper, and to 1cm
% if using A4 paper. (This probably isn't strictly necessary.)
% If using another size paper, use default 1cm margins.
\ifthenelse{\lengthtest { \paperwidth = 11in}}
    { \geometry{top=.5in,left=.5in,right=.5in,bottom=.5in} }
    {\ifthenelse{ \lengthtest{ \paperwidth = 297mm}}
        {\geometry{top=2cm,left=1cm,right=1cm,bottom=1cm} }
        {\geometry{top=2cm,left=1cm,right=1cm,bottom=1cm} }
    }

% Page foot and header
% \pagestyle{empty} % Turn off header and footer
\usepackage{fancyhdr}
\pagestyle{fancy}
\renewcommand{\sectionmark}[1]{\markright{#1}}
\fancyhf{} % remove footer header
\lhead{MCF Algorithms Cheat Sheets}
\chead{\thepage}
\rhead{\rightmark}
\renewcommand{\headrulewidth}{0.4pt}  % remove header line
\renewcommand{\footrulewidth}{0pt}  % default is 0pt

% Redefine section commands to use less space
\makeatletter
\renewcommand{\section}{\@startsection{section}{1}{0mm}%
                                {-1ex plus -.5ex minus -.2ex}%
                                {0.5ex plus .2ex}%x
                                {\normalfont\Large\bfseries}}
\renewcommand{\subsection}{\@startsection{subsection}{2}{0mm}%
                                {-1explus -.5ex minus -.2ex}%
                                {0.5ex plus .2ex}%
                                {\normalfont\normalsize\bfseries}}
\renewcommand{\subsubsection}{\@startsection{subsubsection}{3}{0mm}%
                                {-1ex plus -.5ex minus -.2ex}%
                                {1ex plus .2ex}%
                                {\normalfont\small\bfseries}}
\makeatother

% Define BibTeX command
\def\BibTeX{{\rm B\kern-.05em{\sc i\kern-.025em b}\kern-.08em
    T\kern-.1667em\lower.7ex\hbox{E}\kern-.125emX}}

% Don't print section numbers
\setcounter{secnumdepth}{0}
\setcounter{tocdepth}{1}

\setlength{\parindent}{0pt}
\setlength{\parskip}{0pt plus 0.5ex}

\begin{document}

\title{$3$-dimensional Continued Fraction Algorithms Cheat Sheets}
\author{Sébastien Labbé\footnote{
Université de Liège,
B\^at. B37 Institut de Math\'ematiques,
Grande Traverse 12,
4000 Li\`ege,
Belgium,
\texttt{slabbe@ulg.ac.be}.}}
\date{}

\maketitle



\thispagestyle{empty} % remove page number from first page (after maketitle)


\begin{multicols}{2}
\begin{abstract}

Multidimensional Continued Fraction Algorithms are generalizations of the Euclid
algorithm and find iteratively the gcd of two or more numbers. They are defined
as linear applications on some cone of $\mathbb{R}^d$. In these Cheat Sheets, we
consider $3$-dimensional continued fraction algorithms that acts symmetrically
on the positive cone $\mathbb{R}^d_+$. We include well-known and old ones
(Poincar\'e, Brun, Selmer, Fully Subtractive) and new ones
(Arnoux-Rauzy-Poincar\'e, Reverse, Cassaigne). 

For each algorithm, a cheat sheet gathers a handful of informations most of them
generated with the open source software Sage with the optional Sage package
\texttt{slabbe-0.2.spkg}. The information includes the $n$-cylinders, density
function of an absolutely continuous invariant measure, domain of the natural
extension, lyapunov exponents as well as some aspects regarding combinatorics on
words and digital geometry.

The document ends with a table of comparison of Lyapunov exponents and gives the
code allowing to reproduce any of the results or figures appearing in these
cheat sheets.

\end{abstract}

\columnbreak
\tableofcontents
\end{multicols}

\raggedright

\newpage

\begin{multicols}{3}
% multicol parameters
% These lengths are set only within the two main columns
%\setlength{\columnseprule}{0.25pt}
\setlength{\premulticols}{1pt}
\setlength{\postmulticols}{1pt}
\setlength{\multicolsep}{1pt}
\setlength{\columnsep}{2pt}
\newcommand{\note}[1]{\hfill\textrm{\textcolor{gray}{#1}}}
\newcommand{\args}[1]{\textit{\textcolor{blue}{#1}}}
\newcommand{\stdout}[1]{\textcolor{Sepia}{#1}}
\footnotesize

\input{sections.tex}
\end{multicols}

\footnotesize
\input{lyapunov_table.tex}

\begin{multicols}{3}
% multicol parameters
% These lengths are set only within the two main columns
%\setlength{\columnseprule}{0.25pt}
\setlength{\premulticols}{1pt}
\setlength{\postmulticols}{1pt}
\setlength{\multicolsep}{1pt}
\setlength{\columnsep}{2pt}
\newcommand{\note}[1]{\hfill\textrm{\textcolor{gray}{#1}}}
\newcommand{\args}[1]{\textit{\textcolor{blue}{#1}}}
\newcommand{\stdout}[1]{\textcolor{Sepia}{#1}}
\raggedright
\footnotesize

\section{Sage Code}
\begin{refsegment}
\subsection{Requirements}
\begin{verbatim}
sage -v
sage -i slabbe-0.2.spkg
tikz2pdf
\end{verbatim}
\subsection{Definition}
\begin{verbatim}
sage: import slabbe.mult_cont_frac as mcf
sage: algo = mcf.Brun()
\end{verbatim}
\subsection{Invariant measure}
\begin{verbatim}
sage: algo.invariant_measure_plot(n_iterations=10^6,
                                  ndivs=40,
                                  norm='1')
\end{verbatim}
\subsection{Density function}
This section is hand written.
\subsection{Cylinders}
\begin{verbatim}
sage: cocycle = algo.matrix_cocycle()
sage: s = cocycle.tikz_n_cylinders(3, scale=3)
sage: view(s, tightpage=True)
\end{verbatim}
\subsection{Natural extension}
\begin{verbatim}
sage: n_iterations = 1200
sage: marksize = .8
sage: s = algo.natural_extension_tikz(n_iterations, 
                                      marksize=marksize,
                                      group_size="2 by 2")
sage: view(s, tightpage=True)
\end{verbatim}
\subsection{Lyapunov exponents}
\begin{verbatim}
sage: ntimes=10 
sage: n_iterations=10^6
sage: algo.lyapunov_exponents_sample(ntimes, n_iterations)
                          min       mean      max       std
+-----------------------+---------+---------+---------+---------+
  $\theta_1$              0.3013    0.3046    0.3063    0.00093
  $\theta_2$              -0.1131   -0.1122   -0.1113   0.00038
  $1-\theta_2/\theta_1$   1.3674    1.3683    1.3694    0.00056
\end{verbatim}
\subsection{Substitutions}
\begin{verbatim}
sage: algo.substitutions()
\end{verbatim}
\subsection{$S$-adic word example}
\begin{verbatim}
sage: v = (1,e,pi)
sage: it = algo.coding_iterator(v)
sage: [next(it) for _ in range(10)]
sage: algo.s_adic_word(v)
sage: map(w[:10000].number_of_factors, range(21))  
\end{verbatim}
\subsection{Discrepancy}
\begin{verbatim}
TODO
\end{verbatim}
\subsection{Dual substitutions}
\begin{verbatim}
sage: D = algo.dual_substitutions()
\end{verbatim}
\subsection{E one star}
\begin{verbatim}
sage: v = (1,e,pi)
sage: n = 8
sage: P = algo.e_one_star_patch(v, n)
sage: s = P.plot_tikz()
sage: view(s, tightpage=True)
\end{verbatim}
\subsection{Matrices}
\begin{verbatim}
sage: cocycle = algo.matrix_cocycle()
sage: cocycle.gens()
\end{verbatim}
\end{refsegment}
\printbibliography[segment=8,
heading=subbibliography,
title={References}]
\newpage

\section*{Acknowledgments}

This work is part of the project ``Dynamique des algorithmes du pgcd : une
approche Algorithmique, Analytique, Arithmétique et Symbolique (Dyna3S)''
(ANR-13-BS02-0003)  supported by the Agence Nationale de la Recherche. The
author is supported by a postdoctoral Marie Curie fellowship (BeIPD-COFUND)
cofunded by the European Commission. I wish to thank Valérie Berthé, Pierre
Arnoux, Vincent Delecroix and Thierry Monteil for many discussions on the
experimental aspects of MCF algorithms.



% You can even have references
\rule{0.3\linewidth}{0.25pt}
\scriptsize
\bibliographystyle{plain}
\bibliography{biblio}

\end{multicols}
\end{document}


