\documentclass[10pt,landscape,a4paper]{article}

%*****************************************************************************
% Copyright (C) 2014 Volker Braun <vbraun.name@gmail.com> (git-cheat-sheet.tex)
% Copyright (C) 2015 Sébastien Labbé (MCFAlgorithms cheat sheets)
% Creative Commons Attribution-Share Alike 3.0 License.
%*****************************************************************************

\usepackage{multicol}
\usepackage{calc}
\usepackage{ifthen}
\usepackage[landscape]{geometry}
\usepackage{amsmath,amsthm,amsfonts,amssymb}
\usepackage{color,graphicx,overpic}
\usepackage{hyperref}
\usepackage[T1]{fontenc}
\usepackage{lmodern}         % Polices vectorielles
\usepackage[utf8]{inputenc}  % Codage UNICODE (UTF-8)

\pdfinfo{
  /Title (MCF Algorithms cheat sheet.pdf)
  /Creator (TeX)
  /Producer (pdfTeX 1.40.0)
  /Author (Sébastien Labbé)
  /Subject (Example)
  /Keywords (pdflatex, latex,pdftex,tex)}

% This sets page margins to .5 inch if using letter paper, and to 1cm
% if using A4 paper. (This probably isn't strictly necessary.)
% If using another size paper, use default 1cm margins.
\ifthenelse{\lengthtest { \paperwidth = 11in}}
    { \geometry{top=.5in,left=.5in,right=.5in,bottom=.5in} }
    {\ifthenelse{ \lengthtest{ \paperwidth = 297mm}}
        {\geometry{top=2cm,left=1cm,right=1cm,bottom=1cm} }
        {\geometry{top=2cm,left=1cm,right=1cm,bottom=1cm} }
    }

% Page foot and header
% \pagestyle{empty} % Turn off header and footer
\usepackage{fancyhdr}
\pagestyle{fancy}
\renewcommand{\sectionmark}[1]{\markright{#1}}
\fancyhf{} % remove footer header
\lhead{MCF Algorithms Cheat Sheets}
\chead{\thepage}
\rhead{\rightmark}
\renewcommand{\headrulewidth}{0.4pt}  % remove header line
\renewcommand{\footrulewidth}{0pt}  % default is 0pt

% Redefine section commands to use less space
\makeatletter
\renewcommand{\section}{\@startsection{section}{1}{0mm}%
                                {-1ex plus -.5ex minus -.2ex}%
                                {0.5ex plus .2ex}%x
                                {\normalfont\Large\bfseries}}
\renewcommand{\subsection}{\@startsection{subsection}{2}{0mm}%
                                {-1explus -.5ex minus -.2ex}%
                                {0.5ex plus .2ex}%
                                {\normalfont\normalsize\bfseries}}
\renewcommand{\subsubsection}{\@startsection{subsubsection}{3}{0mm}%
                                {-1ex plus -.5ex minus -.2ex}%
                                {1ex plus .2ex}%
                                {\normalfont\small\bfseries}}
\makeatother

% Define BibTeX command
\def\BibTeX{{\rm B\kern-.05em{\sc i\kern-.025em b}\kern-.08em
    T\kern-.1667em\lower.7ex\hbox{E}\kern-.125emX}}

% Don't print section numbers
\setcounter{secnumdepth}{0}
\setcounter{tocdepth}{1}

\setlength{\parindent}{0pt}
\setlength{\parskip}{0pt plus 0.5ex}

%My Environments
\newtheorem{example}[section]{Example}
% -----------------------------------------------------------------------

\begin{document}

\title{Multidimensional Continued Fraction Algorithms Cheat Sheets}
\author{Sébastien Labbé}

\maketitle
\thispagestyle{empty} % remove page number from first page (after maketitle)

\begin{multicols}{3}
% multicol parameters
% These lengths are set only within the two main columns
%\setlength{\columnseprule}{0.25pt}
\setlength{\premulticols}{1pt}
\setlength{\postmulticols}{1pt}
\setlength{\multicolsep}{1pt}
\setlength{\columnsep}{2pt}
\newcommand{\note}[1]{\hfill\textrm{\textcolor{gray}{#1}}}
\newcommand{\args}[1]{\textit{\textcolor{blue}{#1}}}
\newcommand{\stdout}[1]{\textcolor{Sepia}{#1}}

\tableofcontents

\raggedright
\footnotesize

%\columnbreak
\newpage

\input{sections.tex}
\end{multicols}

\footnotesize
\input{lyapunov_table.tex}

\begin{multicols}{3}
% multicol parameters
% These lengths are set only within the two main columns
%\setlength{\columnseprule}{0.25pt}
\setlength{\premulticols}{1pt}
\setlength{\postmulticols}{1pt}
\setlength{\multicolsep}{1pt}
\setlength{\columnsep}{2pt}
\newcommand{\note}[1]{\hfill\textrm{\textcolor{gray}{#1}}}
\newcommand{\args}[1]{\textit{\textcolor{blue}{#1}}}
\newcommand{\stdout}[1]{\textcolor{Sepia}{#1}}
\raggedright
\footnotesize

\section{Sage Code}
\begin{refsegment}
\subsection{Requirements}
\begin{verbatim}
sage -v
sage -i slabbe-0.2.spkg
tikz2pdf
\end{verbatim}
\subsection{Definition}
\begin{verbatim}
sage: import slabbe.mult_cont_frac as mcf
sage: algo = mcf.Brun()
\end{verbatim}
\subsection{Invariant measure}
\begin{verbatim}
sage: algo.invariant_measure_plot(n_iterations=10^6,
                                  ndivs=40,
                                  norm='1')
\end{verbatim}
\subsection{Density function}
This section is hand written.
\subsection{Cylinders}
\begin{verbatim}
sage: cocycle = algo.matrix_cocycle()
sage: s = cocycle.tikz_n_cylinders(3, scale=3)
sage: view(s, tightpage=True)
\end{verbatim}
\subsection{Natural extension}
\begin{verbatim}
sage: n_iterations = 1200
sage: marksize = .8
sage: s = algo.natural_extension_tikz(n_iterations, 
                                      marksize=marksize,
                                      group_size="2 by 2")
sage: view(s, tightpage=True)
\end{verbatim}
\subsection{Lyapunov exponents}
\begin{verbatim}
sage: ntimes=10 
sage: n_iterations=10^6
sage: algo.lyapunov_exponents_sample(ntimes, n_iterations)
                          min       mean      max       std
+-----------------------+---------+---------+---------+---------+
  $\theta_1$              0.3013    0.3046    0.3063    0.00093
  $\theta_2$              -0.1131   -0.1122   -0.1113   0.00038
  $1-\theta_2/\theta_1$   1.3674    1.3683    1.3694    0.00056
\end{verbatim}
\subsection{Substitutions}
\begin{verbatim}
sage: algo.substitutions()
\end{verbatim}
\subsection{$S$-adic word example}
\begin{verbatim}
sage: v = (1,e,pi)
sage: it = algo.coding_iterator(v)
sage: [next(it) for _ in range(10)]
sage: algo.s_adic_word(v)
sage: map(w[:10000].number_of_factors, range(21))  
\end{verbatim}
\subsection{Discrepancy}
\begin{verbatim}
TODO
\end{verbatim}
\subsection{Dual substitutions}
\begin{verbatim}
sage: D = algo.dual_substitutions()
\end{verbatim}
\subsection{E one star}
\begin{verbatim}
sage: v = (1,e,pi)
sage: n = 8
sage: P = algo.e_one_star_patch(v, n)
sage: s = P.plot_tikz()
sage: view(s, tightpage=True)
\end{verbatim}
\subsection{Matrices}
\begin{verbatim}
sage: cocycle = algo.matrix_cocycle()
sage: cocycle.gens()
\end{verbatim}
\end{refsegment}
\printbibliography[segment=8,
heading=subbibliography,
title={References}]
\newpage


% You can even have references
\rule{0.3\linewidth}{0.25pt}
\scriptsize
\bibliographystyle{plain}
\bibliography{biblio}

\end{multicols}
\end{document}


