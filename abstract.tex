\begin{abstract}
Multidimensional Continued Fraction Algorithms are generalizations of the
Euclid algorithm and find iteratively the gcd of two or more numbers. They are
defined as linear applications on some subcone of $\mathbb{R}^d$. We consider
multidimensional continued fraction algorithms that acts symmetrically on the
positive cone $\mathbb{R}^d_+$ for $d=3$. We include well-known and old ones
(Poincar\'e, Brun, Selmer, Fully Subtractive) and new ones
(Arnoux-Rauzy-Poincar\'e, Reverse, Cassaigne). 

For each algorithm, one page (called cheat sheet) gathers a handful of
informations most of them generated with the open source software Sage
\cite{sage} with the optional Sage package \texttt{slabbe-0.2.spkg}
\cite{labbe_slabbe_2015}. The information includes the $n$-cylinders, density
function of an absolutely continuous invariant measure, domain of the natural
extension, lyapunov exponents as well as data regarding combinatorics on
words, symbolic dynamics and digital geometry, that is, associated
substitutions, generated $S$-adic systems, factor complexity, discrepancy,
dual substitutions and generation of digital planes.

The document ends with a table of comparison of Lyapunov exponents and gives the
code allowing to reproduce any of the results or figures appearing in these
cheat sheets.
\end{abstract}
