\section{Requirements}
The image in these cheat sheets were created with the optional Sage package
\texttt{slabbe-0.2.spkg}, the script \texttt{tikz2pdf} and Sage with the
following version:
\begin{verbatim}
sage -v
SageMath Version 6.10.beta3, Release Date: 2015-11-05
\end{verbatim}
My Sage package is available at:
\begin{verbatim}
sage -p slabbe-0.2.spkg TODO
tikz2pdf TODO
\end{verbatim}
Define a Multidimensional Continued Fraction algorithm:
\begin{verbatim}
sage: from slabbe.mult_cont_frac import Brun
sage: algo = Brun()
\end{verbatim}
You may replace \texttt{Brun} above by any of the following:
\begin{verbatim}
Brun, Poincare, Selmer, FullySubtractive, 
ARP, Reverse, Cassaigne
\end{verbatim}
\section{Sage Code}
This section shows how to reproduce any of the results in these Cheat Sheets.
\subsection{Definition}
This section is hand written.
\subsection{Invariant measure}
\begin{verbatim}
sage: algo.invariant_measure_plot(n_iterations=10^6,
                 ndivs=40, norm='1')
\end{verbatim}
\subsection{Density function}
This section is hand written.
\subsection{Cylinders}
\begin{verbatim}
sage: cocycle = algo.matrix_cocycle()
sage: s = cocycle.tikz_n_cylinders(3, scale=3)
sage: view(s, tightpage=True)
\end{verbatim}
\subsection{Natural extension}
\begin{verbatim}
sage: s = algo.natural_extension_tikz(n_iterations=1200, 
               marksize=.8, group_size="2 by 2")
sage: view(s, tightpage=True)
\end{verbatim}
\subsection{Lyapunov exponents}
The algorithm that computes Lyapunov exponents was provided to me
by Vincent Delecroix, in June 2013. I translated his C code into cython.
\begin{verbatim}
sage: algo.lyapunov_exponents_table(n_orbits=10, n_iterations=10^6)
\end{verbatim}
\subsection{Substitutions}
\begin{verbatim}
sage: algo.substitutions()
\end{verbatim}
\subsection{$S$-adic word example}
\begin{verbatim}
sage: v = (1,e,pi)
sage: it = algo.coding_iterator(v)
sage: [next(it) for _ in range(10)]
sage: algo.s_adic_word(v)
sage: map(w[:10000].number_of_factors, range(21))  
\end{verbatim}
\subsection{Discrepancy}
\begin{verbatim}
TODO
\end{verbatim}
\subsection{Dual substitutions}
\begin{verbatim}
sage: D = algo.dual_substitutions()
\end{verbatim}
\subsection{E one star}
\begin{verbatim}
sage: v = (1,e,pi)
sage: P = algo.e_one_star_patch(v, n=8)
sage: s = P.plot_tikz()
sage: view(s, tightpage=True)
\end{verbatim}
\subsection{Matrices}
\begin{verbatim}
sage: cocycle = algo.matrix_cocycle()
sage: cocycle.gens()
\end{verbatim}
\newpage
